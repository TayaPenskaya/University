\documentclass{amsart}

\usepackage[T2A]{fontenc}
\usepackage[utf8]{inputenc}
\usepackage[russian]{babel}

\usepackage{amsmath}

\begin{document}
\selectlanguage{russian}

\section{Введение}

Data Mining - один из шагов в knowledge discovery in databases.
\begin{itemize}
    \item Сбор данных 
    \item Выделение признаков
    \item Применение алгоритмов машинного обучения
\end{itemize}

Data Analysis
\begin{itemize}
    \item Exploratory DA - невооруженным алгоритмами взглядом пытаемся понять закономерности в данных
    \item Confirmatory DA - выдвигаем гипотезы, пытаемся подтвердить
    \item Perdictory DA
    \item Визуализация данных
\end{itemize}

Data Science 
\begin{itemize}
    \item Сбор данных
    \item Интеграция данных (data integration)
    \item Хранение данных (data warehousing)
    \item Анализ данных
    \item Высокопроизводительные вычисления (high-performance computing)
\end{itemize}

Machine learning - чисто алгоритмы, которые мы потом хотим где-то применить для решении проблемы.

\section{Обучение с учителем}

Задача индукционного обучения: найти закономерность по небольшому числу известных нам фактов, чтобы в дальнейшем обобщить ее на все возможные в дальшейшем ситуации.

\subsection{Формулировка задачи}
\hfill \newline
X - множество объектов; \\
Y - множество меток (ответов); \\
$y: X \rightarrow Y$ - неизвестная целевая функция (зависимость).

Про часть объектов что-то знаем:
$D = \{(x_{i}, y_{i})\}$ - размеченный набор данных, где $\{x_{1}, .. , x_{|D|}\} 	\subset X$ - объекты, а $y_{i} = y(x_{i})$ - известные метки (значения целевой функции).

Нужно найти алгоритм a: $X \rightarrow Y$ решающую (классифицирующую) функцию, приближающую целевую y на X.


\subsection{Объекты}
\hfill \newline
$f_{j}: X -> D_{j}$ - признаки объектов. \\
Типы признаков:
\begin{itemize}
    \item бинарный - $\{0,1\}$
    \item категориальный - $D_{j}$ конечно (цвет)
    \item порядковый - $D_{j}$ конечно и упорядочено (сорт муки)
    \item численный - $\mathbb{R}$ (длина)
\end{itemize}

Объект - вектор значений признаков этого объекта $(f_{1}(x), .. , f_{n}(x))$.

Очень плохо преобразовывать напрямую категориальные в численные, так как численные сравниваются по расстоянию между ними, а это может не совпадать с разницей у категориальных (пример: 1 - blue, 2 - yellow, 3 - orange -- расстояния между 1,2 и 2,3 одинаковые, но по факту это не так).

\subsection{Ответы}
\begin{itemize}
    \item Классификация
        \begin{itemize}
            \item $Y = \{-1, +1\}$ - бинарная классификация (любит ли человек бургер)
            \item $Y = \{1, .. , M\}$ - выбор из M непересекающихся классов (самое любимое блюдо)
            \item $Y = \{0,1\}^{M}$ - для каждого из M пересекающихся классов выбрать свою метку (гражданином каких стран человек является)
        \end{itemize}
    \item Ранжирование
        \begin{itemize}
            \item Y - конечно (частично) упорядоченное множество (ранжирование по предпочтительности)
        \end{itemize} 
    \item Регрессия
        \begin{itemize}
            \item $Y = \mathbb{R}$ или Y = $\mathbb{R}^{m}$ (с какой вероятностью человек посетит конкретную страну / все страны мира)
        \end{itemize}
\end{itemize}

\subsection{Как найти алгоритм a?}
\hfill \newline
Предсказательная модель - параметрическое семейство отображений $A = \{M(x, \theta)| \theta \in \Theta\}$, где $M: X \times \Theta -> Y$ - некоторая функция (зафиксировали $\theta$, по входящему x получаем некоторое y), а $\Theta$ - множество возможных значений параметра $\theta$.\\ \\ 
Если будем рассматривать полиномы 1ой степени, то получаем линейную модель, где $\Theta$ будет n-мерным вектором $\mathbb{R}^{n}$, где n - число признаков.\\ \\
Метод обучения - отображение из мн-ва датасетов в мн-во алгоритмов.
Зафиксировали модели (хотим найти лучшую из них), пришел датасет, мы хотим, чтобы метод обучения по датасету выбрал лучшую модель.

Метод обучения:
\begin{itemize}
    \item Валидационный метод - как ставим задачу
    \item Модель обучения - что за множество алгоритмов, из которого мы выбираем
\end{itemize}

\subsection{Насколько хорошо a приближаюет y?}
\hfill \newline
Функция потерь - величина ошибки алгоритма a на объекте x.
\begin{itemize}
    \item классификация - угадали или нет 
    \item регрессия - насколько сильно угадали (расстояние - обычно, квадратичная функция потерь)
\end{itemize}

Чтобы оценить качество алгоритма, эмпирический риск - среднее по ошибкам по всем объектам.

\subsection{Проблема переобучения}
\hfill \newline
Проблема переобучения — начиная с определенного уровня сложности предсказательной модели, чем лучше алгоритм показывает себя на тренировочном наборе данных D, тем хуже он работает на реальных объектах.

\end{document}