\documentclass{amsart}

\usepackage[T2A]{fontenc}
\usepackage[utf8]{inputenc}
\usepackage[russian]{babel}

\usepackage[table,xcdraw]{xcolor}

\usepackage{float}

\usepackage{graphics}

\begin{document}
\selectlanguage{russian}

\section{Требования к базам данных}

Структура базы данных == Схема базы данных
\begin{itemize}
  \item Что хранится
  \item Как хранится
  \item Права доступа
\end{itemize}

\subsection{Типы данных}
\begin{itemize}
  \item Простые (числа, строки)
  \item Структурированные (адрес, телефон)
  \item Сущность (студент, группа) - любой различимый объект, который может быть представлен в базе данных. Между сущностями могут быть связи (студент учится в группе).
\end{itemize}
 Сущности отличаются от структурированных данных тем, что равные сущности не обязательно указывают на одно и то же (два студента с одинаковыми инициалами).

\subsection{Ограничение целостности}
\hfill \\
Ограничения целостности представляют собой ограничения, налагаемые на
значения, которые разрешено принимать некоторой переменной, или комбинации переменных.
\begin{itemize}
  \item На отдельные значение (возраст не может быть равен -3)
  \item На записи (если запись состоит из начала и конца, хотим, чтобы конец всегда был больше начала)
  \item На наборы записей (номера паспортов)
  \item На связи данных (студент учится в одной группе)
\end{itemize}

\section{Развитие баз данных}

\subsection{Простые и Структурированные файлы}
\hfill

\begin{table}[h!]
\resizebox{\columnwidth}{!}{%
\begin{tabular}{|l|l|l|}
\hline
\rowcolor[HTML]{C0C0C0} 
{\color[HTML]{000000} Модель данных} & {\color[HTML]{000000} Структура}                                                                                           & {\color[HTML]{000000} Замечания}                                                              \\ \hline
простой файл                       & \begin{tabular}[c]{@{}l@{}}Заголовок (названия столбцов)\\ Данные (просто текст)\end{tabular}                              & \begin{tabular}[c]{@{}l@{}}Чтобы что-то найти, \\ придется прочитать все данные.\end{tabular} \\ \hline
структур. файл                      & \begin{tabular}[c]{@{}l@{}}Заголовок (названия столбцов, типы и длины)\\ Данные (записи одинаковой структуры)\end{tabular} & \begin{tabular}[c]{@{}l@{}}Теперь зная номер записи, \\ мы можем к ней перейти.\end{tabular}  \\ \hline
\end{tabular}%
}
\end{table}

Достоинства
\begin{itemize}
    \item Простота чтения 
\end{itemize}

Недостатки
\begin{itemize}
    \item Сложность поиска (особенно, если нужно найти строки, удовлетворяющему какому-то нетривиальному критерию (к примеру, на несколько элементов данных))
    \item Сложность обработки
    \item Сложность изменения
    \item Сложность хранения данных разных типов
    \item Нет проверки целостности
\end{itemize}

\subsection{Более интересные модели}

\begin{table}[h!]
\resizebox{\columnwidth}{!}{%
\begin{tabular}{|l|l|l|}
\hline 
\rowcolor[HTML]{C0C0C0} 
Модель                                                               & Представление данных                                                                                                                                                                                              & Замечания                                                                                                                                                                                                                                                                                                                                                                                                  \\ \hline
\begin{tabular}[c]{@{}l@{}}Файловые \\ системы\end{tabular}          & \begin{tabular}[c]{@{}l@{}}Файл – одна запись\\ Каталоги – подчиненные записи\\ \\ Пример:\\ Иванов И.И./Оценки/Java – 4\end{tabular}                                                                             & \begin{tabular}[c]{@{}l@{}}Достоинства:\\  - Структурирование данных\\  - Простота реализации\\ Недостатки:\\  - Сложно извлекать требуемые данные\\  - Нет проверки целостности\\  - Большое количество файлов\end{tabular}                                                                                                                                                                               \\ \hline
\rowcolor[HTML]{EFEFEF} 
\begin{tabular}[c]{@{}l@{}}Иерархические \\ базы данных\end{tabular} & \begin{tabular}[c]{@{}l@{}}Храним дерево записей, \\ для каждой записи знаем \\ ее тип и структуру (т.е. можем \\ это проверять)\\ Отношения родитель – ребенок\end{tabular}                                      & \begin{tabular}[c]{@{}l@{}}Недалеко ушло от файловой системы, \\ но хранится в меньшем числе файлов\\ и следит за корректной структурой \\ всех записей.\\ Достоинства:\\  - Проверка целостности записей и отношений\\  - Последовательное расположение записей\\  - Эффективность реализации\\ Недостатки:\\  - Представление только древовидных данных\\  - Нет отношения многие-ко-многим\end{tabular} \\ \hline
\begin{tabular}[c]{@{}l@{}}Сетевые\\ базы данных\end{tabular}        & \begin{tabular}[c]{@{}l@{}}Ориентированный граф записей\\ Отношения владелец – запись\\ У записи может быть более чем\\ один владелец (к примеру, у оценки \\ может быть владелец предмет и студент)\end{tabular} & \begin{tabular}[c]{@{}l@{}}Достоинства\\  - Представление всех типов связей\\  - Возможность описания структуры\\  - Эффективность реализации\\ Недостатки\\  - Сложность реализации\\  - Жесткое ограничение структуры\end{tabular}                                                                                                                                                                       \\ \hline
\rowcolor[HTML]{EFEFEF} 
\begin{tabular}[c]{@{}l@{}}Реляционные\\ базы данных\end{tabular}    & \begin{tabular}[c]{@{}l@{}}Данные хранятся в таблицах\\ Проверка целостности заданных связей\\ Связи задаются в запросах\end{tabular}                                                                             & \begin{tabular}[c]{@{}l@{}}Достоинства\\  - Представление всех типов связей\\  - Гибкая структура данных\\  - Математическая модель\\ Недостатки\\  - Сложность реализации\\  - Сложность представления \\ иерархических данных\\  - Сложность составления \\ эффективных запросов\end{tabular}                                                                                                            \\ \hline
\begin{tabular}[c]{@{}l@{}}Объектные\\ базы данных\end{tabular}      & \begin{tabular}[c]{@{}l@{}}Сущность – объект\\ Связь – поле\\ Ограничения целостности – \\ определение объекта\end{tabular}                                                                                       & \begin{tabular}[c]{@{}l@{}}Достоинства\\  - Простота представления объектов\\  - Гибкая структура данных\\  - «Логичное» направление ссылок\\ Недостатки\\  - Сложность реализации\\  - Сложность миграции схемы\\  - Малая распространенность\end{tabular}                                                                                                                                                \\ \hline
\rowcolor[HTML]{EFEFEF} 
NoSQL                                                                & \begin{tabular}[c]{@{}l@{}}Ориентированы на специфичные задачи, \\ но в общем случае проигрывают\end{tabular}                                                                                                     & \begin{tabular}[c]{@{}l@{}}Типы\\  - Документ-ориентированные\\  - Ключ-значение\\  - Табличные и столбчатые\\  - Графовые\end{tabular}                                                                                                                                                                                                                                                                    \\ \hline
\end{tabular}%
}
\end{table}

\hfill \\ \\

\section{Архитектура РСУБД}

Программа выдает запрос, обращается к хранилищу данных, получает ответ.
Но понятно, что хранилище само не может выдавать ответ, что же между ними? \\
Отдельно есть база данных, к которой может присоединяться большое кол-во программ. \\
Со стороны программы есть \textbf{драйвер}, которому она передает запрос. Со стороны бд есть собственный \textbf{драйвер}, который умеет декодировать полученный запрос от драйвера программы. К \textbf{СУБД} пришел запрос, который обрабатывается с помощью \textbf{разборщика запроса} (парсит запрос) -  у разборщика знает только схему базы данных (что есть сущность Оценки, к примеру). Далее с помощью \textbf{построителя плана исполнения} смотрим на статистику по данным, узнал, сколько есть памяти, и сказал, как мы будем выполнять этот запрос. Потом все переходит в руки \textbf{исполнителя запроса}.

\end{document}